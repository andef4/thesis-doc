Almost all interpretability methods work on image classification tasks. To learn how these methods are applied, how they work and what output they generate, the first step in this work is applying these methods on a classification task. We chose a dataset from the medical imaging field: The NIH (National Institutes of Health, United States) chest x-ray dataset.

We downloaded the dataset, trained a neural network on the dataset and applied the selected methods described above.

\section{NIH Chest X-ray dataset}


\section{Model Training}


\subsection{Inception Resnet v2}

\nblink{nhs-chest-xray/inception\_resnetv2.ipynb}

\begin{itemize}
    \item  chest x-ray, sample data set (small subset)
    \item  multi label classification
    \item  current state of the art networkj: inception  resetnet v2 (accurate + fast to train on single GPU) \cite{todo}
    \item  using learnings and code from project2
    \item  slightly modified inception resnet v2 implementation (grayscale instead of color)
    \item  did not work, convert to RGB and downscale images (offline, preprocess.ipynb)
    \item  results: 40\% validation accuracy => bad
    \item smaller batch size (citation needed) 5, => not better
    \item next: use full dataset
    \item download 42 gigs of data
    \item simple python multiprocessing for preprocessing
\end{itemize}

\nblink{nhs-chest-xray/preprocess.ipynb}

\subsection{ResNet}
\nblink{nhs-chest-xray/resnet.ipynb}

Instead of waiting for the expected very long learning time for Inception-Resnet V2, we did a short test with Resnet50 with pretrained parameters (ImageNet) on the sample dataset. Only the last layer is unfreezed.

\begin{itemize}
    \item Epoch time of ~40 seconds instead of (TODO) 10 minutes/3 hours for full dataset
    \item resnet50 with pretrained
    \item resnet50 without pretrained
    \item resnet18 without pretrained
    \item resnet18 without pretrained full dataset
\end{itemize}

\subsection{Only images with actual diagnosis}
https://www.kaggle.com/kmader/train-simple-xray-cnn 
For testing out the methods, images with findings are much more helpful, so filter them out.

Better results:


\subsection{Single label only}
65\%

\subsection{Densenet}
\nblink{nhs-chest-xray/densenet.ipynb}



% https://medium.com/@jrzech/reproducing-chexnet-with-pytorch-695ff9c3bf66

Loading the saved model did not work, because the used version of PyTorch was old and incompatible with ours.

=> Retraining

\subsubsection{Image resize problems}
In the Densenet code there was a reference that some images were not properly resized. We did a short check on our preprocessed data, but it revelead no such problems. Used bash command:

\begin{minted}{bash}
find . -name '*.png' -exec file {} \; | cut -c 37-45 | uniq
\end{minted}

\subsection{Densenet single}
\nblink{nhs-chest-xray/densenet\_singleclass.ipynb}
\nblink{nhs-chest-xray/inception\_resnetv2\_singleclass.ipynb}

\begin{itemize}
    \item Fail: Not correctly setting up classes (last layer has 1000 classes for imagenet, we have 14)
    \item Fail 2: Train and validation set do not have same classes because of ordering in dataset loading code
\end{itemize}

After training for a short time with the fixes, we get an acceptable accuracy rate of 55\%. We then started to calculate the accuracy per class, as has been done on the reproduce-chexnet discussed above. We quickly discovered that only the "No findings" class was accurate, an all other classes had zero correct classifications.

\begin{itemize}
    \item test with no findings removed => not that better, finds only 3 classes correctly
    \item try with weighted cross entropy loss
\end{itemize}
