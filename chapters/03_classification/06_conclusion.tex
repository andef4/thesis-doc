\section{Discussion}
Comparing all three algorithms, there is no clear outcome which one performs the best.

From the black box methods, RISE looks to perform better, especially when looking at more example images in the linked Jupyter notebooks. It seems like
the superpixels generated by LIME do not work that well for this dataset, because the diseases do not show up as distinct uniform patches, but rather as faint color changes.

The adjustable superpixel count for the LIME output is problematic. Using only a small superpixel count works well when the network only looks at a small part of an image, but when it looks
at a big part, this will not show up until we use more superpixels. The size of the output should be decided by the method and not by the user of the method, because the user can only guess what
the correct superpixel count could be.

Grad-CAM has a strange problem where it does not generate a heat map for some images. A reason for this behaviour could not be found. If there is output,
the quality is quite high, but the marked regions are very big. In the Infiltration case (Figures \ref{lime_example_3}, \ref{rise_example_3}, \ref{grad_cam_example_3}), Grad-CAM delivers by far
the best output, especially compared to RISE.

A big advantage of Grad-CAM is its performance, which it has in common with all white box methods. The image has to go through the neural network only once, recording activations and gradients in the process which are then analyzed in the end. In comparison, black box methods have to run many modified images through the neural network, not just one.

\section{Conclusion}
Grad-CAM provides some good insights into the neural network and will be modified for image segmentation.

RISE performs better than LIME and does not use a configuration variable which is difficult to correctly set and will therefore be prioritized for the image segmentation modifications.
