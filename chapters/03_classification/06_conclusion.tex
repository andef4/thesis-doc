\section{Discussion}
Comparing all three algorithms, there is no clear picture which one performs the best.

From the black box methods, RISE looks to perform better, especially when looking at more example images in the linked jupyter notebooks. It seems like
the superpixels generated by LIME do not work that well for this dataset, because the diseases do not show up as distinct uniform patches, but rather as faint color changes.

Grad-CAM has a strange problem where it does not generate a heat map for some images. A reason for this behaviour could not be found. If there is output,
the quality is quite high, but the marked regions are very big. In the Infiltration case (Figures \ref{lime_example_3}, \ref{rise_example_3}, \ref{grad_cam_example_3}, Grad-CAM delivers by far
the best output, especially compared to RISE.

A big advantage of Grad-CAM is its performance, which it has in common with all white box methods. The image only has to go trough the neural network once, recording activations and gradients in the process which are then analyzed in the end. In comparison, black box methods have to run many modified images trough the neural network, not just one.

\section{Conclusion}
Grad-