\chapter{Methods}
\section{Lime}
LIME\cite{todo} stands for X

Interprets models as blackboxes. Covers parts of the image with a mask.

Examples for PyTorch, unknown simple porting to pytorch would be => TODO investiage


\section{RISE}
Similar technology as LIME, uses PyTorch. Newer than LIME, paper asserts to be better.

Written for PyTorch





\section{Methods}

\begin{tabular}{|l|l|l|l|}
\hline
 \textbf{Method} & \textbf{Model blackbox} & \textbf{Portable to PyTorch} & \textbf{Usable for image segmentation} \\ \hline
 RISE & Yes & Supported out of the box & Yes\\ \hline
 LIME & Yes & Easy & Yes \\ \hline
 Layer-wise Relevance Propagation (Heatmapping.org) & No & Should be possible & ? \\ \hline
 Deconvolution (Zeiler 2014) \\ \hline
 DeepLIFT
 PetternNet
 Deeo Taylor
 Grad-CAM
 
\end{tabular}

PatternNet: https://arxiv.org/pdf/1705.05598.pdf
SHAP: https://github.com/slundberg/shap

heatmap presentation:
Baehrens'10 Gradient
Sundarajan'17 Int Grad
Zintgraf'17 Pred Diff
Ribeiro'16 LIME
Haufe'15 Pattern
Symonian'13 Gradient
Zeiler'14 Occlusions
Fong'17 M Perturb
Zurada'94 Gradient
Poulin'06 Additive
Zeiler'14 Deconv
Lundberg'17 Shapley
Landecker'13 Contrib Prop
Caruana'15 Fitted Additive
Bazen'13 Taylor
Springenberg'14 Guided BP
Montavon'17 Deep Taylor
Bach'15 LRP
Zhou'16 GAP
Kindernnans'17 PatternNet
Shrikumar'17 DeepLIFT
Zhang'16 Excitation BP
Selvaraju'17 Grad-CAM
