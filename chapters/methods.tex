\chapter{Methods}
\section{Lime}
LIME\cite{todo} stands for X

Interprets models as blackboxes. Covers parts of the image with a mask.

Examples for PyTorch, unknown simple porting to pytorch would be => TODO investiage


\section{RISE}
Similar technology as LIME, uses PyTorch. Newer than LIME, paper asserts to be better.

Written for PyTorch





\section{Method types}
One of the main questions we have to answer is which methods for interpretability can be used on an Image Segmentation task.
Generally there are two types of methods for interpretability: Blackbox and whitebox.
Blackbox methods like LIME or RISE do not need an understanding of the underlying model and can therefore be used on arbitrary network architectures and even non-deep learning technologies like decision trees.
Whitebox methods need to have access to the underling model, because they analyze a certain part of the network (e.g. extracted features).

\subsection{Blackbox}
Blackbox methods modify the input image, run it trough the network and then analyze how the classification output of the network has changed with the modification.
We think it is feasible to apply this method to image segmentation too, but instead of looking how the classification output changes we check how the segmentation
output changes.

\subsection{Whitebox}
Whitebox methods require access to the architecture of the model and therefore only work on some specific architectures classes.

The most common method for image segmentation in the medical image analysis (todo citation needed) is the UNet\cite{todo} architecture.



https://medium.com/@keremturgutlu/semantic-segmentation-u-net-part-1-d8d6f6005066


\begin{tabularx}{\textwidth}{|l|l|l|}
\hline
\textbf{Method} & \textbf{Model blackbox} & \textbf{Works with PyTorch} \\ \hline
RISE & Yes & Supported out of the box \\ \hline
LIME & Yes & Indepenent of library \\ \hline
Layer-wise Relevance Propagation (LRP) & No & Should be possible  \\ \hline
DeepLIFT
PatternNet
Deep Taylor
Grad-CAM
\hline
\end{tabularx}

PatternNet: https://arxiv.org/pdf/1705.05598.pdf
SHAP: https://github.com/slundberg/shap

heatmap presentation:
Baehrens'10 Gradient
Sundarajan'17 Int Grad
Zintgraf'17 Pred Diff  https://arxiv.org/pdf/1702.04595.pdf
Ribeiro'16 LIME
Haufe'15 Pattern
Symonian'13 Gradient
Zeiler'14 Occlusions
Fong'17 M Perturb
Zurada'94 Gradient
Poulin'06 Additive
Zeiler'14 Deconv
Lundberg'17 Shapley
Landecker'13 Contrib Prop
Caruana'15 Fitted Additive
Bazen'13 Taylor
Springenberg'14 Guided BP
Montavon'17 Deep Taylor
Bach'15 LRP
Zhou'16 GAP
Kindernnans'17 PatternNet
Shrikumar'17 DeepLIFT
Zhang'16 Excitation BP
Selvaraju'17 Grad-CAM
