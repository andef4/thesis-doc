\section{Medical background}
The brain scans in the BraTS 2018 dataset were made with MRI scanners. The brain is mostly consists mostly of water. Water molecules contain two hydrogen molecules




The scans in the BraTS 2018 datasets are created with four different settings of the MRI scanner:

GD-enhancing tumor (ET — label 4),
peritumoral edema (ED — label 2),
necrotic and non-enhancing tumor core (NCR/NET — label 1)

% https://www.researchgate.net/post/What_are_the_differences_between_enhancing_and_nonenhancing_lesions_in_MRI

The scanners detect three different types of kinds of tumor tissues. 
Gadolinium is a chemical compound given during MRI scans that highlights areas of inflammation (Active Lesions). A gadolinium-enhanced  MRI scan shows active lesions, meaning that there is a breakdown of the blood-brain barrier and inflammation is present.


\begin{itemize}
    \item Native (T1)
    \item Post-contrast T1-weighted (T1Gd)
    \item T2-weighted (T2)
    \item T2 Fluid Attenuated Inversion Recovery (FLAIR)
\end{itemize}







* MRI detects water
* hydrogen molecule spinning inside a H2O molecule, randomly
* MRI is a big magnet, making the molecules spinning in a specific direction (North-South or South-North)

Parallel vs anti parallel direction
* Turn proton, stop magnet, measure how long it takes until proton is in its resting position again (releases electromagnetic energy which can be measures) => depends on material. Measure with 