\section{Medical background}
The scans in the BraTS 2018 datasets are created with four different settings of the MRI scanner:

GD-enhancing tumor (ET — label 4),
peritumoral edema (ED — label 2),
necrotic and non-enhancing tumor core (NCR/NET — label 1)

% https://www.researchgate.net/post/What_are_the_differences_between_enhancing_and_nonenhancing_lesions_in_MRI

The scanners detect three different types of kinds of tumor tissues. 
Gadolinium is a chemical compound given during MRI scans that highlights areas of inflammation (Active Lesions). A gadolinium-enhanced  MRI scan shows active lesions, meaning that there is a breakdown of the blood-brain barrier and inflammation is present.


\begin{itemize}
    \item Native (T1)
    \item Post-contrast T1-weighted (T1Gd)
    \item T2-weighted (T2)
    \item T2 Fluid Attenuated Inversion Recovery (FLAIR)
\end{itemize}