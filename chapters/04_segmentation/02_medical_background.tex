\section{Medical background}
The brain scans in the BraTS 2018 dataset were made with MRI scanners\cite{mriscanner}.
The brain consists mostly of water. Water molecules contain two hydrogen molecules, each of which has one proton. A proton spins in a specific direction. When a strong magnet is applied to a proton, the proton starts to spin in the same direction as the magnet. When the magnet is deactivated, the protons go back into their resting position. 

A MRI scanner can generate a very precise and targeted magnetic field. Depending on the material a proton is part of (gray matter, necrotic tissue etc.), the resting position of the proton is different and a different amount of energy is released when the proton returns to its resting position. This

The scanners detect three different types of kinds of tumor tissues. 
Gadolinium is a chemical compound given during MRI scans that highlights areas of inflammation (Active Lesions). A gadolinium-enhanced  MRI scan shows active lesions, meaning that there is a breakdown of the blood-brain barrier and inflammation is present.

Gadolinium-enhancing tumor (ET — label 4),
Peritumoral edema (ED — label 2),
Necrotic and non-enhancing tumor core (NCR/NET — label 1)

\begin{itemize}
    \item Native (T1)
    \item Post-contrast T1-weighted (T1Gd)
    \item T2-weighted (T2)
    \item T2 Fluid Attenuated Inversion Recovery (FLAIR)
\end{itemize}
