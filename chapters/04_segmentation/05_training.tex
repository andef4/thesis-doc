\section{Training}
\nblink{brats/04\_basic\_unet.ipynb}

The first try training the U-Net model returned very bad accuracy values event after training for a long time. To avoid the lengthy process with evaluating different network architectures and parameters like in the classification model for the NIH Chest X-Ray dataset, we started to read different resources for enhancements of the basic U-Net architecture

The first change we implemented was correctly normalizing the input data by using a transform from the PyTorch library.
The second idea we foundOne of the first things we found was introduction a batch normalization layer after every convolutional layer.



first try. baaad
second try with normalization: no change
build evaluation
third try with batch norm: much higher loss that slowy goes down, evaluate still shows random pixels
5. run it for some hours, it worked!


