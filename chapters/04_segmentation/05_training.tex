\section{Training}
\nblink{brats/04\_basic\_unet.ipynb}
The detailed results of the training sessions are available in the results directory of the GitHub repository: \href{https://github.com/andef4/thesis-code/tree/master/brats/results/}{brats/results}.

The first try training the U-Net model showed good loss values soon after starting the training session. When we actually looked at the output of the network, the output was completely blank.

To avoid the lengthy process with evaluating different network architectures and parameters like in the classification model for the NIH Chest X-Ray dataset, we started to read different resources for enhancements of the basic U-Net architecture

The first change we implemented was correctly normalizing the input data by using a transform from the PyTorch library.
Retraining after this change showed similar low loss values as in the first attempt, with the same result when evaluating the network output - a blank image.

The second change we did after some investigation was the introduction of a batch normalization layer after every convolutional layer. The first training run after this change showed much worse loss numbers which decreased slowly.

We implemented the calculation of the Dice score (also called the F1 score) and the Hausdorff distance (see chapter \ref{hausdorff_distance_chapter} for an explanation) in addition to the loss value to have some more insight into the training process.

We let the training process run for around 4 hours. After that, the output images from the network looked very promising.
