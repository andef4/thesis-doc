\section{BraTS 2018 dataset}
\section{Medical background}
The scans in the BraTS 2018 datasets are created with four different settings of the MRI scanner:

GD-enhancing tumor (ET — label 4),
peritumoral edema (ED — label 2),
necrotic and non-enhancing tumor core (NCR/NET — label 1)

% https://www.researchgate.net/post/What_are_the_differences_between_enhancing_and_nonenhancing_lesions_in_MRI

The scanners detect three different types of kinds of tumor tissues. 
Gadolinium is a chemical compound given during MRI scans that highlights areas of inflammation (Active Lesions). A gadolinium-enhanced  MRI scan shows active lesions, meaning that there is a breakdown of the blood-brain barrier and inflammation is present.


\begin{itemize}
    \item Native (T1)
    \item Post-contrast T1-weighted (T1Gd)
    \item T2-weighted (T2)
    \item T2 Fluid Attenuated Inversion Recovery (FLAIR)
\end{itemize}

\section{Preprocessing \& Slice selection}
These are also saved as 2 byte integers with the values 0 (not tumor), 1 (todo), 2(todo) and 4(todo).

Every layer is saved as a 2 byte signed integer. We convert these to 1 byte unsigned integers and scale the range to 0-255, so the highest value in the 2 byte signed integer is 255 and the lowest is 0.
This way we generate a grayscale image that can be viewed with a normal image viewer and should be processable by a normal convolutional neural network

\section{Neural network architecture}
https://github.com/milesial/Pytorch-UNet
https://github.com/usuyama/pytorch-unet

modifications: do not use deprectated functions

\section{Training}

first try. baaad
second try with normalization: no change
build evaluation
third try with batch norm: much higher loss that slowy goes down, evaluate still shows random pixels
5. run it for some hours, it worked!

\section{Modifing and applying RISE (Single Pixel)}
\section{Modifing and applying RISE (Multi Pixel)}
\section{Modifing and applying Grad-CAM}
