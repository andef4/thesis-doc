\section{Modifying and applying RISE (Single Pixel)}
\nblink{brats/06\_rise.ipynb}

The approach how to apply RISE on a segmentation task the same as with Grad-CAM in the last chapter. Pixels in the output of a segmentation model are equivalent to classes of a classification model.
As a first version, we only analyze a single pixel. We chose the first pixel in the tumor ground truth segment by iterating over the segment linearly until the first active pixel is found.


\subsection{Results}

\begin{figure}[H]
\centering
\includegraphics[width=8cm]{chapters/04_segmentation/images/rise_single_pixel.png}
\caption{Saliency map analyzing the topmost pixel in the scan, overlaid on the generated network output}
\label
\end{figure}



\subsection{Discussion}
looks correct, maybe a bit off => could be scaling

The produced output looks correct, the color blob is at the correct position. It clearly shows that the neural network looks at the correct location to generate the segmentation.
Apart from this basic correctness verification, no further insigt is provided by the saliency map, because the resolution generated by RISE is too low.

\subsection{Conclusion}
The generated output is low resolution but still helpful, we therefore decided to build a version of RISE which works on all pixels of the segmentation.

\section{Modifying and applying RISE (Multi Pixel)}
\nblink{17\_rise\_multipixel.ipynb}

TODO: implementation

\subsection{Results}
\begin{figure}[H]
    \centering
    \begin{subfigure}{.5\textwidth}
        \centering
        \includegraphics[width=\linewidth]{chapters/04_segmentation/images/rise_multipixel_max_1-0.png}
        \caption{ the text for a}
    \end{subfigure}%
    \begin{subfigure}{.5\textwidth}
        \centering
        \includegraphics[width=\linewidth]{chapters/04_segmentation/images/rise_multipixel_max_1-1.png}
        \caption{b}
    \end{subfigure}
    \caption{Explanation text}
\end{figure}

\subsection{Discussion}

\subsection{Conclusion}
