\section{BraTS 2018 dataset}
The University of Pennsylvania hosts the Multimodal Brain Tumor Segmentation (BraTS) Challenge \cite{bratschallenge}  every year. In this contest, participants build algorithms to generate brain tumor segments from four different MRI scan modalities. The participants build and train their algorithms on the provided BraTS \cite{menze2015multimodal} dataset.

We use the 2018 version of the dataset, which contains new scans and revised ground truth segments.

\subsection{Data format}
The data is provided in the NIfTI file format (file ending *.nii.gz). It can be opened by the Python library NiBabel\cite{nibabel} which transforms the data into ordinary Python numpy arrays.

\subsection{Ground truth/labels}
The dataset contains ground truth segments with the following tissue types:

\begin{itemize}
    \item Gadolinium-enhancing tumor (ET - label value 4)
    \item Peritumoral edema (ED - label value 2),
    \item Necrotic and non-enhancing tumor core (NCR/NET - label value 1)
\end{itemize}

The labels are also saved in the NIfTI file format, using the integer values from above for every tissue type.
