\section{RISE}
Like LIME, RISE is a block box method and works on manipulation of the input images. Instead of using superpixels, RISE generates masks that are then applied to the input images

\begin{figure}[H]
    \centering
    \begin{subfigure}[t]{.32\textwidth}
        \centering
        \includegraphics[width=\linewidth]{chapters/02_methods/images/rise/rise_original.png}
        \caption{Original image}
    \end{subfigure}\hfill%
    \begin{subfigure}[t]{.32\textwidth}
        \centering
        \includegraphics[width=\linewidth]{chapters/02_methods/images/rise/rise0_mask.png}
        \caption{RISE mask}
    \end{subfigure}\hfill%
    \begin{subfigure}[t]{.32\textwidth}
        \centering
        \includegraphics[width=\linewidth]{chapters/02_methods/images/rise/rise0_applied.png}
        \caption{RISE mask applied to original image by multiplication}
    \end{subfigure}
    \caption{By multiplication the input image (left) with a RISE mask (center), a modified input is generated.}
    \label{rise_mask0}
\end{figure}


\begin{figure}[H]
    \centering
    \begin{subfigure}[t]{.32\textwidth}
        \centering
        \includegraphics[width=\linewidth]{chapters/02_methods/images/rise/rise_original.png}
        \caption{Original image}
    \end{subfigure}\hfill%
    \begin{subfigure}[t]{.32\textwidth}
        \centering
        \includegraphics[width=\linewidth]{chapters/02_methods/images/rise/rise1_mask.png}
        \caption{RISE mask}
    \end{subfigure}\hfill%
    \begin{subfigure}[t]{.32\textwidth}
        \centering
        \includegraphics[width=\linewidth]{chapters/02_methods/images/rise/rise1_applied.png}
        \caption{RISE mask applied to original image by multiplication}
    \end{subfigure}
    \caption{By multiplication the input image (left) with a RISE mask (center), a modified input is generated. In comparison fo Figure \ref{rise_mask0}, this input image retains much less information.}
    \label{rise_mask1}
\end{figure}

Figure \ref{rise_mask0} and Figure \ref{rise_mask1} show two RISE masks applied to the same input image.



\begin{figure}[H]
\centering
\includegraphics[width=12cm]{chapters/02_methods/images/rise/sheep.png}
\caption{Image from original paper explaining some classes}
\end{figure}


* [ ] doc: RISE: explain heatmap, why is it there? => gaus, upsampling
