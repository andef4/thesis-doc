This chapter compares the goals initially set in the introduction and the requirements document in \autoref{appendix_a}  with the accomplishments of this thesis.

\section{Goals}

\subsubsection{Research existing methods for image classification}
12 methods were evaluated and listed in \autoref{chapter_methods}.

\subsubsection{Asses if the found methods can be modified for image segmentation}
The three methods RISE, LIME and Grad-CAM were selected for modification to work on segmentation tasks. They are explained in detail in \autoref{chapter_methods}.

\subsubsection{Modify the methods for image segmentation}
Two of the methods (RISE and Grad-CAM) were successfully modified to work on image segmentation tasks. The output from RISE was usable, but had low spatial resolution. The output from Grad-CAM did not provide any significant insight into the inner workings of the neural network.  The LIME method was not modified, because it delivered underwhelming results on the NIH Chest X-ray dataset. A new method based on ideas proposed in 2014 by Zeiler et al. \cite{zeiler2014visualizing} was developed and named Hausdorff Distance Masks (HDM).

\subsubsection{Build and train a neural network on the BraTS brain tumor segmentation dataset}
A neural network with the U-Net architecture was trained on the BraTS dataset and delivered good accuracy values.

\subsubsection{Apply the modified methods on the trained neural network}
The two modified methods RISE and Grad-CAM and the newly developed Hausdorff Distance Masks method were successfully applied on the BraTS dataset.

\subsubsection{Analyze, evaluate and discuss the results}
Every chapter or section which present results also contains a discussion and conclusion section.

\subsubsection{Provide images for teaching material}
The Jupyter notebooks in the git repository contain a large amount of images. A script was written which can extract all images from Jupyter notebok and save them as normal PNG images. The script is described in the \autoref{chapter_infrastructure}.

\subsubsection{Build a reusable Python library}
A Python library which allows the application of the RISE and HDM method was developed, documented and published on GitHub and PyPI.

\section{Conclusion}
All non-optional goals were completed successfully. Sadly, the output from the modified Grad-CAM method was not usable, but the black box RISE method generated good results. Instead of modifying additional white box and black box methods (optional goals), we decided to build a new black box method from scratch called Hausdorff Distance Masks. The new method generated the best results of all methods, especially on the 3D BraTS dataset.
