This chapter compares the goals initially set in the introduction and the requirements document in Appendix A with the accomplishments of this thesis.

\subsubsection{Research existing methods for image classification}
A list of methods is listed in chapter 2, three methods (RISE, LIME and Grad-Cam) are explained in detail in the same chapter.

\subsubsection{Asses if the found methods can be modified for image segmentation}
The methods listed in chapter 2 were asseset and the three methods listed above were selected for modification.

\subsubsection{Modify the methods for image segmentation}
Two of the methods (RISE and Grad-CAM) were successfully modified to work on image segmentation tasks.
The LIME method was not modified, because it delivered underwhelming results on the NIH Chest X-ray dataset.
A new method based on ideas proposed in 2014 by Zeiler et al. was developed and named Hausdorff distance masks (HDM for short).

\subsubsection{Build and train a neural network on the BraTS brain tumor segmentation dataset}
A neural network with the U-Net architecture was trained on the BraTS dataset and delivered good accuracy values.

\subsubsection{Apply the modified methods on the trained neuronal network}
The two modified methods RISE and Grad-CAM and the newly developed Hausdorff distance mask method were successfully applied on the BraTS dataset.

\subsubsection{Analyze, evaluate and discuss the results}
Every chapter or section which shows some kind of result also contains a discussion and conclusion section.

\subsubsection{Provide images for the teaching materials of Mauricio Reyes}
The Jupyter notebooks in the git repository contain a large amount of images. A script was written which can extract all images from Jupyter notebok
and save them as normal PNG images.

\subsection{Build a reusable Python library}
A Python library which allows the application of the RISE and HDM method was developed, documented and published on GitHub and PyPI.
