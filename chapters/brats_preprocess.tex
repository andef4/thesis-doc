\chapter{BraTS 2018 preprocessing}
The scans in the BraTS 2018 datasets are created with four different settings of the MRI scanner:

\begin{itemize}
    \item Native (T1)
    \item Post-contrast T1-weighted (T1Gd)
    \item T2-weighted (T2)
    \item T2 Fluid Attenuated Inversion Recovery (FLAIR)
\end{itemize}

Every layer is saved as a 2 byte signed integer. We convert these to 1 byte unsigned integers and scale the range to 0-255, so the highest value in the 2 byte signed integer is 255 and the lowest is 0.
This way we generate a grayscale image that can be viewed with a normal image viewer and should be processable by a normal convolutional neural network

\section{Segments}
The scanners detect three different types of kinds of tumor tissues. These are also saved as 2 byte integers with the values 0 (not tumor), 1 (todo), 2(todo) and 4(todo).