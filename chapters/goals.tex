\chapter{Goals}

\section{Brain tumors}
The first thing we want to learn is the medical background of the thesis, answering the following questions:
\begin{itemize}
    \item What are brain tumors?
    \item What types of brain tumors exists?
    \item How do they look like on a scan? What scanner types exists (MRI/PET/CT)? Which one is used when?
    \item How does a scan look like? Colors? Layers? 2D/3D images?
    \item How does a doctor analyze such a scan?
\end{itemize}

\section{The BraTS dataset}
\begin{itemize}
    \item What scanner is used and what tumor types are detected?
    \item What data format is used to save the scans?
    \item In what format is the labeling saved?
    \item Build a loader for the dataset
    \item Display some scans
\end{itemize}

\section{Neural Network}

\begin{itemize}
    \item Research current state of the art networks for image segmentation
    \item Investigate U-Net architecture
    \item Build and train network for the BraTS dataset in PyTorch
    \item Investigate and implement data augmentation for the dataset
    \item Implement automatic hyperparameter optimization (e.g. Grid Search)
    \item Train network on BFH or University of Bern Cluster
\end{itemize}

\section{LIME}

\section[RISE}


\section{Visualization Tools}

\section{Library}
Build a library with a common interface for the implemented tools, so that all of them can be applied easily.
\begin{itemize}
    \item Write example programs for the library to use with PyTorch
    \item (Optional) Write example programs for the library to use with TensorFlow
    \item Write documentation how to use the library
    \item (Optional) Publish on conda
    \item (Optional) Publish on PyPI
    
\end{itemize}

\section{GANT Chart}
