\label{chapter_infrastructure}
\section{Infrastructure}
All the work for this thesis was done on a consumer PC, consisting of an AMD Ryzen 2700X processor, 16 GB of memory and an NVIDIA GeForce RTX 2080.

Most of the programming work has taken place using Jupyter notebooks. The Jupyter notebook server was password protected and exposed to the internet with port forwarding set up on the internet router. This enabled remote work from a train or school while still having access to the powerful desktop hardware required to train neural network models.

\section{Script to extract images from Jupyter notebooks}

We wrote a Python Script which extracts images from all Jupyter notebooks inside a specified folder. This simplified the process to extract images for this document and also allows the constituent to easily access images which he can use in his teaching material.

The script is available on GitHub: \href{https://github.com/andef4/thesis-code/blob/master/extract_notebook_images.py}{extract\_notebook\_images.py}.

\section{Code style}
The code style of the Python code is checked using the flake8 program. It is setup as a git precommit hook. This way, the code style is always checked when we try to make a commit. Most code for this thesis is inside Jupyter notebooks which flake8 can not parse to check the code style. The solution is a script which uses the Jupytext \cite{jupytext} program to extract Python code from the Jupyter notebook files and then runs the flake8 program on the extracted source code.

The script is also available on GitHub: \href{https://github.com/andef4/thesis-code/blob/master/ipynb_lint.sh}{ipynb\_lint.sh}.
